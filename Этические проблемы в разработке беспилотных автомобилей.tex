\documentclass{article}
\usepackage[utf8]{inputenc}
\usepackage[russian]{babel}
\usepackage{hyperref}

\begin{document}

\title{Этические проблемы в разработке беспилотных автомобилей}

\author{Эрнисов Айдар Эрнисович}
\maketitle

\section{Введение}

Беспилотные автомобили, основанные на искусственном интеллекте, несут в себе обещание революционизировать транспорт, повысив безопасность, эффективность и удобство. Однако их внедрение сопряжено с рядом сложных этических вопросов, которые необходимо тщательно изучить и решить, прежде чем эти автомобили станут повсеместной реальностью.

\section{Тезисы}

\subsection{Безопасность и принятие рисков}

\begin{itemize}
    \item Главная идея: Обеспечение безопасности беспилотных автомобилей является первостепенной задачей. Это включает в себя разработку надежных систем ИИ, способных безопасно маневрировать в различных дорожных условиях, а также минимизацию рисков отказов оборудования и кибератак.
    
    \item Структура:
    \begin{itemize}
        \item Просмотреть потенциальные угрозы безопасности, такие как аварии, взломы и сбои в работе систем.
        \item Сравнить уровень безопасности беспилотных автомобилей с традиционными автомобилями.
        \item Просмотреть этические вопросы, связанные с тестированием и развертыванием беспилотных автомобилей на дорогах общего пользования.
        \item Просмотреть этические вопросы, связанные с тестированием и развертыванием беспилотных автомобилей на дорогах общего пользования.
    \end{itemize}
\end{itemize}


\subsection{Автономия и ответственность}

\begin{itemize}
    \item Главная идея:  Определить, кто несет ответственность в случае аварии с беспилотным автомобилем: производитель, разработчик программного обеспечения, владелец или пассажиры?
    
    \item Структура:
    \begin{itemize}
        \item Проанализировать различные правовые и этические рамки для определения ответственности в автономных системах.
        \item Рассмотреть вопросы прозрачности и подотчетности в алгоритмах принятия решений ИИ, используемых в беспилотных автомобилях.
        \item Рассмотреть потенциальные проблемы, связанные с делегированием ответственности за управление транспортным средством машине.
    \end{itemize}
\end{itemize}


\subsection{Этика приоритетов и права пешеходов}

\begin{itemize}
   \item Главная идея: Как беспилотные автомобили должны принимать решения в сложных ситуациях, когда необходимо выбирать между причинением вреда пешеходам, пассажирам или другим участникам дорожного движения?

   \item Структура:

   \begin{itemize}
        \item Рассмотреть различные этические принципы, которые могут быть использованы для программирования беспилотных автомобилей (например, утилитаризм, деонтология).
        \item Рассмотреть проблемы справедливости и неравенства в алгоритмах принятия решений ИИ, используемых в беспилотных автомобилях.
        \item Просмотреть анализы потенциального влияния беспилотных автомобилей на права и безопасность пешеходов.
    \end{itemize}

\end{itemize}


\subsection{Прозрачность и конфиденциальность данных}

\begin{itemize}
    \item Главная идея: Беспилотные автомобили будут собирать большие объемы данных о своих пассажирах, окружающей среде и других участниках дорожного движения. Как эти данные будут собираться, храниться и использоваться?
    
    \item Структура:
    \begin{itemize}
        \item Просмотреть проблемы конфиденциальности и защиты данных, связанные с беспилотными автомобилями.
        \item Просмотреть потенциальные риски использования данных беспилотных автомобилей для слежки или дискриминации.
        \item Просмотреть анализы необходимость разработки четких правил и норм для обеспечения этичного сбора и использования данных беспилотных автомобилей.
    \end{itemize}
\end{itemize}


\section{Заключение}



\bibliography{bibliography}

\end{document}
